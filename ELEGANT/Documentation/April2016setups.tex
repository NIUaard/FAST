\documentclass[notitlepage,twocolumn,nofootinbib,showpacs,preprintnumbers,superscriptaddress,amsmath,amssymb]{revtex4-1}
%\documentclass[notitlepage,nofootinbib,showpacs,preprintnumbers,superscriptaddress,amsmath,amssymb,12pt]{revtex4-1}

% Some other (several out of many) possibilities
%\documentclass[preprint,aps]{revtex4}
%\documentclass[preprint,aps,draft]{revtex4}
%\documentclass[prb]{revtex4}% Physical Review B

\usepackage{graphicx}% Include figure files
\usepackage{dcolumn}% Align table columns on decimal point
\usepackage{bm}% bold math
\usepackage{amsmath}
\usepackage{multirow}
%\usepackage{algorithm}

\newcommand{\astra}{{\sc Astra }}
\newcommand{\elegant}{{\sc Elegant }}
\newcommand{\astragenerator}{{\sc AstraGenerator }}
\newcommand{\mafia}{{\sc Mafia }}
\newcommand{\rmsemit}{\mbox{$\tilde{\varepsilon}$}}
\newcommand{\mean}[1]{\mbox{$\langle{#1}\rangle$}}
\newcommand{\matlab}{{\sc Matlab }}
\newcommand{\ud}{\mathrm{d}}

\usepackage[colorlinks=true]{hyperref}

%\nofiles
% MATH -----------------------------------------------------------
\newcommand{\norm}[1]{\left\Vert#1\right\Vert}
\newcommand{\abs}[1]{\left\vert#1\right\vert}
\newcommand{\set}[1]{\left\{#1\right\}}
\newcommand{\V}{\overrightarrow}
\newcommand{\N}{\hat}
\newcommand{\G}{\nabla}
\newcommand{\D}{\partial}
\newcommand{\Real}{\mathbb R}
\newcommand{\Int}{\mathbb N}
\newcommand{\eps}{\varepsilon}
\newcommand{\To}{\longrightarrow}
\newcommand{\BX}{\mathbf{B}(X)}
\newcommand{\A}{\mathcal{A}}
\newcommand{\re}{\mathcal{Re}}
\newcommand{\im}{{\cal Im}}
\newcommand{\T}{\mathbb{T}}
\newcommand{\dop}{\delta_{0p}}
\newcommand{\dom}{\delta_{0m}}
%


\usepackage{listings}

% Python style for highlighting
\newcommand\pythonstyle{\lstset{
        language=Python,
%        basicstyle=\ttm,
        otherkeywords={self},             % Add keywords here
%        keywordstyle=\ttb\color{deepblue},
        emph={MyClass,__init__},          % Custom highlighting
 %       emphstyle=\ttb\color{deepred},    % Custom highlighting style%
%        stringstyle=\color{deepgreen},
        frame=tb,                         % Any extra options here
        showstringspaces=false            % 
}}


% Python environment
\lstnewenvironment{python}[1][]
{
    \pythonstyle
    \lstset{#1}
}
{}


%\setlength{\oddsidemargin}{-3mm} \setlength{\evensidemargin}{-3mm}
\begin{document}
\title{Lattice settings for FAST}
%\thanks{ This work was partially sponsored by the the DOE contracts DE-SC00????? to Northern Illinois University 
%and DE-AC02-07CH11359 to the Fermi Research Alliance, LLC which operates Fermilab. }
%\author{P.  Piot} \affiliation{Northern Illinois Center for
%Accelerator \& Detector Development and Department of Physics,
%Northern Illinois University, DeKalb IL 60115,
%USA} \affiliation{Accelerator Physics Center, Fermi National
%Accelerator Laboratory, Batavia, IL 60510, USA}
%\author{M. Andorf} \affiliation{Northern Illinois Center for
%Accelerator \& Detector Development and Department of Physics,
%Northern Illinois University, DeKalb IL 60115,
%USA} \author{V. A. Lebedev} \affiliation{Accelerator Physics Center, Fermi National
%Accelerator Laboratory, Batavia, IL 60510, USA}
%\author{S. Chattopadhyay} \affiliation{Northern Illinois Center for
%Accelerator \& Detector Development and Department of Physics,
%Northern Illinois University, DeKalb IL 60115,
%USA} \affiliation{Accelerator Physics Center, Fermi National
%Accelerator Laboratory, Batavia, IL 60510, USA}
\date{\today}
	
\begin{abstract}
This Note summarizes possible setups for the {\sc fast} injector. The beam-dynamics calculations especially focus on the ``commissioning" scenarii with single-bunch charges of 200, 20, and 1 pC. 
\end{abstract}
\preprint{Vers. 1 -- chic. off}
\maketitle
%
\section{Introduction}

Operating point for the gun should be 40MV/m. We should check the dark current dependence on field and solenoids strength. 

\subsection{Overview}


\subsection{Aperture and magnets constraints in the injector}

\subsection{Choice of quadrupole-family ``knobs"  \& matching points}
Throughout this note we take Q118, Qxxx, and Qyyy to be always turned off as this quadrupole magnets were initially skewed to add the capability of producing  flat beams~\cite{jun}. 

We take the following approach for matching the beam, given a set of input beam parameters downstream of CAV2. 

When the {\bf compressor is off}, the quadrupole magnets are powered to produce an acceptable betatron function at the $s=cc$~m corresponding to then location of an interaction point (location of the channeling-radiation crystal in the first running period). Achieving a betratron function on the order of 2~m at this location enable full transmission but also provide acceptable betratron function at the location of the dispersive screen X124 to allow for a precise relative momentum-spread measurement~\cite{francoisIPAC13}. 

The case when the {\bf compressor is turned on} is lightly different. In such a case the first quadrupole-magnet  telescope (Q222, Q232, Q123, Q123) is used to achieve a small horizontal (bending-plane) betatron function between the D123 and D234 while maintaining the vertical betatron function to acceptable values. The matching criterion for the horizontal betatron function is commonly used in bunch compressor as it was shown to minimize bending-plane emittance dilution due to energy-spread generation during the compression process (via space charge or coherent-synchrotron radiation effects). 


\section{Beamline settings}
%
\subsection{Overview}

\subsection{Nominal charge $Q=200$~pC}
\subsubsection{Original injector optimization}

\subsubsection{Injector optimization based on achieved machine parameters (week of May 26th, 2016)}
In May 2016, it was realize that CC2 will not be operated at average accelerating voltage much larger than $\bar{V}\simeq 15$~MV/m. Consequently the beamline was reoptimized given this limitation and a possible set of value is presented in Table.~\ref{tab:ctrlrmOptim}. 
%\subsection{Reduced charge $Q=20$~pC}
%\subsection{Ultralow-emittance mode $Q=1$~pC}

\subsubsection{Simulation of as-operated injector (during the week of May 26th, 2016)}
A main concern during the week of 26, was the ultraviolet laser spot on the photocathode surface which display a strong $x-y$ coupling term. The beamsize were on the order of Xxx~$\mu$m in both direction but the correlation coefficient was close to $r \equiv\frac{\mean{xy}}{\sigma_x\sigma_y}\simeq 0.5$. A macroparticle distribution based on the measured UV distribution was used as a starting point in the simulations; the net results is a slight emittance asymmetry which appear acceptable and not strongly affecting the final emittance achievable at FAST.  




%\section{ 50-MeV beam transport to low-energy dump with bunch compressor off}
%
%Over the last funding period important development have emerged: $(i)$ an integrated simulation of the optical stochastic cooling was implemented~\cite{IPAC16Matt} , $(ii)$ a robust design of a 2~$\mu$m amplifier was proposed~\cite{IPAC15Matt} and $(iii)$  an undulator magnet was acquired. We provide details on these developments and they implications on the readiness of our team to carry out experimental tests of subsystems relevant to the proposed OSC experiment at IOTA. 
%
%\subsection{Overview}

%\subsection{Nominal charge $Q=200$~pC}

%\subsection{Reduced charge $Q=20$~pC}

%\subsection{Ultralow-emittance mode $Q=1$~pC}

\newpage
\appendix
\section{Emittance measurement tool V05302016}

\subsection{Introduction}
An emittance acquisition script {\tt QuadScan.py} was written to assist with performing quadrupole-magnet scan to measure the emittance. The prupose of the script is to $(i)$ acquire and achieve the raw data associated to beam images while the quadrupole magnet is being scanned and $(ii)$ provide, on-line, a limited set processed data. In its current implementation the on-line processed data consists of a list of rms beam size computed from a Gaussian and Super-Gaussian fits, a statistical RMS analysis is also available but was decoupled from the acquisition script as it significantly slowed down the process. The script run on the {\tt clx} cluster. To log onto the cluster use the command {\tt ssh -t -Y outback.fnal.gov launch auto VIASSH}. This appendix provides a description of program along with instructions for running this script.
%
\subsection{Acquisition}
The script {\tt QuadScan.py} acquires image from the selected video server using the {\tt imget()} function of {\tt acnet.py} toolbox. The acquisition sequence is as summarize in the following sequence.

\begin{python}
for i in quadsettings:
   for j in NumberOfImage:
      turn on photocathode laser 
      take & save picture -> Img
      turn off photocathode laser
      take & save picture -> Bkg
   compute X[j]=Img[j]-Bkg[j]
   analysis & display statistics of X[j]
\end{python}
The program does not adjust any other parameter (e.g. number of laser pulses impinging the cathode). As of 5/31/2016, 
the quadrupole magnet is scanned to stay on hysteresis following the conventional procedure used at FAST. In such a 
approach the quadrupole setting is reach starting from zero in increasing (resp. decreasing) steps for positive (resp. negative) 
set points. 

\subsection{On-line analysis}
To expedite the emittance measurement, {\tt QuadScan.py} evaluate the beam sizes for each images. During the week of 5/26/2016 a 
recursive statistical analysis of the image for a variable region of interest (ROI) was implemented. The algorithm was significantly 
slowing down the acquisition process so it was deactivated starting 5/30/2016. Instead the {\tt QuadScan.py} now provides the rms obtained 
from a Gaussian fit of the form 
\begin{eqnarray}
P(u)=b+P_0\times \exp \left( -\frac{(u-\bar{u})^2}{2\sigma_u^2}\right), 
\end{eqnarray}
where $u=[x,y]$, where $\sigma_u$ the standard deviation, $b$ the offset, $P_0$ the peak value and $\bar{u}$ the mean position are fit parameters.  Given the generally non-Gaussian character of the measured distributions we have also implemented a super-Gaussian fit 
with a profile given by
\begin{eqnarray}
P(u)=b+P_0\times \exp \left( -\frac{|u-\bar{u}|^n}{2 w_u^n}\right), 
\end{eqnarray}
where the fit parameters are similar to the ones used for the Gaussian fit. Here the standard deviation is related to the width parameter
$w_u$ as $\sigma_u=w_u\left[2\frac{\Gamma(3/m)}{\Gamma(1/m)}\right]^{1/2}$. 

\subsection{Instruction for running {\tt QuadScan.py}}
%
The script is located in {\tt /home/piot/QuadScan} and can be ran from this directory. 
\begin{itemize}
\item Identify the quadrupole magnet you want to scan and OTR/YAG screen you want to record the image from.
\item View the image on the {\tt imagetool} software and note the video server used. 
\item Go to the directory where script is installed: {\tt cd \~{}piot/QuadScan}. 
\item Open the script QuadScan.py with your favorite editor: {\tt nedit QuadScan.py} and set the following variable to the desired values
\begin{itemize}
\item {\tt srv = 4}, the current video server used to visualize the screen in {\tt imagetool}.
\item {\tt Num  = 21}, the number of setting for the quadrupole being scanned between the maximum and minimum current values. 
\item {\tt Navg = 1}, the number of image taken for each quadrupole-magnet settings.
\item {\tt Quad = 'N:Q110'}, the ACNET address of the quadrupole being scanned. 
\item {\tt Screen = 'X111'}, the screen being use to measure the beam image (this is currently used for naming the image file as the screen is selected by {\tt imagetool}).
\item {\tt Imin =  -3.5}, the lower limit of for the quadrupole magnet current. 
\item {\tt Imax =   3.5}, the upper limit of for the quadrupole magnet current. 
\end{itemize}
\item Stop live acquisition of the {\tt imagetool} software
\item Execute the script: {\tt python QuadScan.py} 
\item For each setting of the quadrupole magnet, the software will display a window showing the average image and its fitted profile (only rms of Gaussian and super-Gaussian fits are display on the  on-line software), upon inspection of the image:
\begin{itemize}
\item Close the graphical window (double click on the upper left square).
\item Enter {\tt y} to accept the image. If the image is corrupted or the number of saturated pixel is too high you might want to reject (enter {\tt n}) the image. In such a case the script will retake the image(s) for the current quadrupole-magnet setting. Before entering {\tt y} or {\tt n} you may want to start {\tt imagetool} to inspect the image (do not forget to stop the {\tt imagetool} live acquisition before continuing with {\tt QuadScan.py}.
\end{itemize}
\item At the end of the scan a editor window will pop-up with some information which include: the location where the raw and pre-processed data have been save, and a dump of the machine status (solenoid, charge, ...). You can enter comment in the header of this file for further use. \end{itemize}
By default the data are saved in the directory {\tt /usr/local/cbs\_files/cns\_write/acl/data\_files/} under a subdirectory with name starting by {\tt QuadScan/QScan\_\$TODAY\$TIME} where {\tt \$TODAY} and {\tt \$TIME} are respectively the date and time when the {\tt QuadScan.py} script was launched. 

%
\begin{figure}[b]
\begin{center}
 \includegraphics[width=0.49\linewidth]{lamvsE@FAST.pdf}
 \includegraphics[width=0.49\linewidth]{Evsgap2p2um.pdf}
\caption{{\sc Left:} range of wavelength (shaded area) attainable with the electron-beam energies available at {\sc fast} using the U3 undulator. The horizontal dashed lines with associated labels represents the wavelength of lasers available at {\sc fast} and illustrate that all the considered laser media could be coupled to the beam in the U3 undulator. {\sc Right:} required beam energy as a function of undulator's strength parameter to have a undulator-resonant wavelength of 2.2~$\mu$m.\label{fig:U3wave} }
\end{center}
\end{figure}

%
%
%%%%%%%%%%%%%%%%%%%%%%%%%%%%
%
\begin{thebibliography}{99}
\end{thebibliography}

\end{document}

